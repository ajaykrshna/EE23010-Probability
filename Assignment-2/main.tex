\documentclass[journal,11pt,onecolumn]{IEEEtran}
\usepackage{setspace}
\usepackage{gensymb}
\singlespacing
\usepackage[cmex10]{amsmath}
\usepackage{amsthm}
\usepackage{mathrsfs}
\usepackage{txfonts}
\usepackage{stfloats}
\usepackage{bm}
\usepackage{cite}
\usepackage{cases}
\usepackage{subfig}
\usepackage{longtable}
\usepackage{multirow}
\usepackage{enumitem}
\usepackage{mathtools}
\usepackage{tikz}
\usepackage{circuitikz}
\usepackage{verbatim}
\usepackage[breaklinks=true]{hyperref}
\usepackage{tkz-euclide} % loads  TikZ and tkz-base
\usepackage{listings}
\usepackage{color}    
\usepackage{array}    
\usepackage{longtable}
\usepackage{calc}     
\usepackage{multirow} 
\usepackage{hhline}   
\usepackage{ifthen}   
\usepackage{lscape}     
\usepackage{chngcntr}
\usepackage{float}
\usepackage{gvv}

\begin{document}

\vspace{3cm}
\author{Ajay Krishnan K\\EE22BTECH11003}

\title{Assignment 2}
\maketitle

\textbf{Question 12.13.6.16}
\begin{enumerate}
    \item Bag I contains 3 red and 4 black balls and Bag II contains 4 red and 5 black balls.
          One ball is transferred from Bag I to Bag II and then a ball is drawn from Bag II.
          The ball so drawn is found to be red in colour. Find the probability that the
          transferred ball is black.
\end{enumerate}

\solution

\let\negmedspace\undefined
\let\negthickspace\undefined
\documentclass[journal,11pt]{IEEEtran}
\usepackage{cite}
\usepackage{amsmath,amssymb,amsfonts,amsthm}
\usepackage{algorithmic}
\usepackage{graphicx}
\usepackage{textcomp}
\usepackage{xcolor}
\usepackage{txfonts}
\usepackage{listings}
\usepackage{enumitem}
\usepackage{mathtools}
\usepackage{gensymb}
\usepackage[breaklinks=true]{hyperref}
\usepackage{tkz-euclide} % loads  TikZ and tkz-base
\usepackage{listings}
\usepackage{gvv}

\begin{document}

\vspace{3cm}
\author{Ajay Krishnan K\\EE22BTECH11003}

\title{Assignment 2}
\maketitle

\textbf{Question 12.13.6.16}
Bag I contains 3 red and 4 black balls and Bag II contains 4 red and 5 black balls.
One ball is transferred from Bag I to Bag II and then a ball is drawn from Bag II.
The ball so drawn is found to be red in colour. Find the probability that the
transferred ball is black.
\end{document}
Thus,
\begin{align}
    P(E=0) & = \frac{3}{7} \\
    P(E=1) & = \frac{4}{7}
\end{align}

Probability that the drawn ball is red,

When the ball being transferred is red,
\begin{align}
    P\brak{\cond{X=0}{E=0}} & = \frac{5}{10} \\
                            & =\frac{1}{2}
\end{align}

When the ball being transferred is black,
\begin{align}
    P\brak{\cond{X=0}{E=1}} & = \frac{4}{10} \\
                            & =\frac{2}{5}
\end{align}

Now the probability of drawn ball being red given the transferred ball
is black is

(According to Bayes' theorem)
\begin{align}
    P\brak{\cond{E=0}{X=1}}
     & = \frac{P\brak{E=0}P\brak{\cond{X=1}{E=0}}}{P\brak{E=1}P\brak{\cond{X=1}{E=1}} + P\brak{E=0}P\brak{\cond{X=1}{E=0}}} \\
     & = \frac{\frac{4}{7} \times \frac{2}{5}}{\frac{3}{7}\times\frac{1}{2}+\frac{4}{7}\times\frac{2}{5}}                   \\
     & = \frac{16}{31}
\end{align}
\end{document}