\documentclass[book,11pt]{IEEEtran}
\usepackage{setspace}
\usepackage{gensymb}
\singlespacing
\usepackage[cmex10]{amsmath}
\usepackage{amsthm}
\usepackage{mathrsfs}
\usepackage{txfonts}
\usepackage{stfloats}
\usepackage{bm}
\usepackage{cite}
\usepackage{cases}
\usepackage{subfig}
\usepackage{longtable}
\usepackage{multirow}
\usepackage{enumitem}
\usepackage{mathtools}
\usepackage{tikz}
\usepackage{circuitikz}
\usepackage{verbatim}
\usepackage[breaklinks=true]{hyperref}
\usepackage{tkz-euclide} % loads  TikZ and tkz-base
\usepackage{listings}
\usepackage{color}    
\usepackage{array}    
\usepackage{longtable}
\usepackage{calc}     
\usepackage{multirow} 
\usepackage{hhline}   
\usepackage{ifthen}   
\usepackage{lscape}     
\usepackage{chngcntr}
\usepackage{float}
\DeclareMathOperator*{\Res}{Res}
\renewcommand\thesection{\arabic{section}}
\renewcommand\thesubsection{\thesection.\arabic{subsection}}
\renewcommand\thesubsubsection{\thesubsection.\arabic{subsubsection}}

\renewcommand\thesectiondis{\arabic{section}}
\renewcommand\thesubsectiondis{\thesectiondis.\arabic{subsection}}
\renewcommand\thesubsubsectiondis{\thesubsectiondis.\arabic{subsubsection}}
\renewcommand\thetable{\arabic{table}}
% correct bad hyphenation here
\hyphenation{op-tical net-works semi-conduc-tor}
\def\inputGnumericTable{}                                 %%

\lstset{
%language=C,
frame=single, 
breaklines=true,
columns=fullflexible
}
%\lstset{
%language=tex,
%frame=single, 
%breaklines=true
%}

\begin{document}
\newtheorem{theorem}{Theorem}[section]
\newtheorem{problem}{Problem}
\newtheorem{proposition}{Proposition}[section]
\newtheorem{lemma}{Lemma}[section]
\newtheorem{corollary}[theorem]{Corollary}
\newtheorem{example}{Example}[section]
\newtheorem{definition}[problem]{Definition}
\newcommand{\BEQA}{\begin{eqnarray}}
\newcommand{\EEQA}{\end{eqnarray}}
\newcommand{\define}{\stackrel{\triangle}{=}}
\bibliographystyle{IEEEtran}
\providecommand{\mbf}{\mathbf}
\providecommand{\pr}[1]{\ensuremath{\Pr\left(#1\right)}}
\providecommand{\qfunc}[1]{\ensuremath{Q\left(#1\right)}}
\providecommand{\sbrak}[1]{\ensuremath{{}\left[#1\right]}}
\providecommand{\lsbrak}[1]{\ensuremath{{}\left[#1\right.}}
\providecommand{\rsbrak}[1]{\ensuremath{{}\left.#1\right]}}
\providecommand{\brak}[1]{\ensuremath{\left(#1\right)}}
\providecommand{\lbrak}[1]{\ensuremath{\left(#1\right.}}
\providecommand{\rbrak}[1]{\ensuremath{\left.#1\right)}}
\providecommand{\cbrak}[1]{\ensuremath{\left\{#1\right\}}}
\providecommand{\lcbrak}[1]{\ensuremath{\left\{#1\right.}}
\providecommand{\rcbrak}[1]{\ensuremath{\left.#1\right\}}}
\theoremstyle{remark}
\newtheorem{rem}{Remark}
\newcommand{\sgn}{\mathop{\mathrm{sgn}}}
\providecommand{\abs}[1]{\left\vert#1\right\vert}
\providecommand{\res}[1]{\Res\displaylimits_{#1}} 
\providecommand{\norm}[1]{\left\lVert#1\right\rVert}
\providecommand{\mtx}[1]{\mathbf{#1}}
\providecommand{\mean}[1]{E\left[ #1 \right]}
\providecommand{\fourier}{\overset{\mathcal{F}}{ \rightleftharpoons}}
\providecommand{\system}[1]{\overset{\mathcal{#1}}{ \longleftrightarrow}}
\newcommand{\solution}{\noindent \textbf{Solution: }}
\newcommand{\cosec}{\,\text{cosec}\,}
\providecommand{\dec}[2]{\ensuremath{\overset{#1}{\underset{#2}{\gtrless}}}}
\newcommand{\myvec}[1]{\ensuremath{\begin{pmatrix}#1\end{pmatrix}}}
\newcommand{\mydet}[1]{\ensuremath{\begin{vmatrix}#1\end{vmatrix}}}
\let\vec\mathbf
\def\putbox#1#2#3{\makebox[0in][l]{\makebox[#1][l]{}\raisebox{\baselineskip}[0in][0in]{\raisebox{#2}[0in][0in]{#3}}}}
     \def\rightbox#1{\makebox[0in][r]{#1}}
     \def\centbox#1{\makebox[0in]{#1}}
     \def\topbox#1{\raisebox{-\baselineskip}[0in][0in]{#1}}
     \def\midbox#1{\raisebox{-0.5\baselineskip}[0in][0in]{#1}}

\vspace{3cm}
Given a triangle with vertices
		\begin{align}
			\label{eq:tri-pts}
			\vec{A} = \myvec{2 \\ -1},\,
			\vec{B} = \myvec{-1 \\ -1},\,
			\vec{C} = \myvec{5 \\ -2}
		\end{align}
\section{Section 1}
\documentclass[journal,11pt,onecolumn]{IEEEtran}
\usepackage{setspace}
\usepackage{gensymb}
\singlespacing
\usepackage[cmex10]{amsmath}
\usepackage{amsthm}
\usepackage{mathrsfs}
\usepackage{txfonts}
\usepackage{stfloats}
\usepackage{bm}
\usepackage{cite}
\usepackage{cases}
\usepackage{subfig}
\usepackage{longtable}
\usepackage{multirow}
\usepackage{enumitem}
\usepackage{mathtools}
\usepackage{tikz}
\usepackage{circuitikz}
\usepackage{verbatim}
\usepackage[breaklinks=true]{hyperref}
\usepackage{tkz-euclide} % loads  TikZ and tkz-base
\usepackage{listings}
\usepackage{color}    
\usepackage{array}    
\usepackage{longtable}
\usepackage{calc}     
\usepackage{multirow} 
\usepackage{hhline}   
\usepackage{ifthen}   
\usepackage{lscape}     
\usepackage{chngcntr}
\usepackage{float}
\usepackage{gvv}

\begin{document}

\vspace{3cm}
\author{Ajay Krishnan K\\EE22BTECH11003}

\title{Assignment 2}
\maketitle

\textbf{Question 12.13.6.16}
\begin{enumerate}
    \item Bag I contains 3 red and 4 black balls and Bag II contains 4 red and 5 black balls.
          One ball is transferred from Bag I to Bag II and then a ball is drawn from Bag II.
          The ball so drawn is found to be red in colour. Find the probability that the
          transferred ball is black.
\end{enumerate}

\solution

\documentclass[journal,11pt,onecolumn]{IEEEtran}
\usepackage{setspace}
\usepackage{gensymb}
\singlespacing
\usepackage[cmex10]{amsmath}
\usepackage{amsthm}
\usepackage{mathrsfs}
\usepackage{txfonts}
\usepackage{stfloats}
\usepackage{bm}
\usepackage{cite}
\usepackage{cases}
\usepackage{subfig}
\usepackage{longtable}
\usepackage{multirow}
\usepackage{enumitem}
\usepackage{mathtools}
\usepackage{tikz}
\usepackage{circuitikz}
\usepackage{verbatim}
\usepackage[breaklinks=true]{hyperref}
\usepackage{tkz-euclide} % loads  TikZ and tkz-base
\usepackage{listings}
\usepackage{color}    
\usepackage{array}    
\usepackage{longtable}
\usepackage{calc}     
\usepackage{multirow} 
\usepackage{hhline}   
\usepackage{ifthen}   
\usepackage{lscape}     
\usepackage{chngcntr}
\usepackage{float}
\usepackage{gvv}

\begin{document}

\vspace{3cm}
\author{Ajay Krishnan K\\EE22BTECH11003}

\title{Assignment 2}
\maketitle

\textbf{Question 12.13.6.16}
\begin{enumerate}
    \item Bag I contains 3 red and 4 black balls and Bag II contains 4 red and 5 black balls.
          One ball is transferred from Bag I to Bag II and then a ball is drawn from Bag II.
          The ball so drawn is found to be red in colour. Find the probability that the
          transferred ball is black.
\end{enumerate}

\solution

\documentclass[journal,11pt,onecolumn]{IEEEtran}
\usepackage{setspace}
\usepackage{gensymb}
\singlespacing
\usepackage[cmex10]{amsmath}
\usepackage{amsthm}
\usepackage{mathrsfs}
\usepackage{txfonts}
\usepackage{stfloats}
\usepackage{bm}
\usepackage{cite}
\usepackage{cases}
\usepackage{subfig}
\usepackage{longtable}
\usepackage{multirow}
\usepackage{enumitem}
\usepackage{mathtools}
\usepackage{tikz}
\usepackage{circuitikz}
\usepackage{verbatim}
\usepackage[breaklinks=true]{hyperref}
\usepackage{tkz-euclide} % loads  TikZ and tkz-base
\usepackage{listings}
\usepackage{color}    
\usepackage{array}    
\usepackage{longtable}
\usepackage{calc}     
\usepackage{multirow} 
\usepackage{hhline}   
\usepackage{ifthen}   
\usepackage{lscape}     
\usepackage{chngcntr}
\usepackage{float}
\usepackage{gvv}

\begin{document}

\vspace{3cm}
\author{Ajay Krishnan K\\EE22BTECH11003}

\title{Assignment 2}
\maketitle

\textbf{Question 12.13.6.16}
\begin{enumerate}
    \item Bag I contains 3 red and 4 black balls and Bag II contains 4 red and 5 black balls.
          One ball is transferred from Bag I to Bag II and then a ball is drawn from Bag II.
          The ball so drawn is found to be red in colour. Find the probability that the
          transferred ball is black.
\end{enumerate}

\solution

\input{tables/main.tex}
Assuming ball is not transferred,
\begin{align}
    P(X=0, Y=0) & = \frac{3}{7} \\
    P(X=1, Y=0) & = \frac{4}{7}
\end{align}

When the ball being transferred is red,
\begin{align}
    P\brak{\cond{X=0,Y=1}{X=0}} & = \frac{5}{10} \\
                                & =\frac{1}{2}
\end{align}

When the ball being transferred is black,
\begin{align}
    P\brak{\cond{X=0,Y=1}{X=1}} & = \frac{4}{10} \\
                                & =\frac{2}{5}
\end{align}

Now the probability of the transferred ball is black given drawn ball being red is

(According to Bayes' theorem)
\begin{align}
    P\brak{\cond{X=1}{X=0,Y=1}}
     & = \frac{P\brak{X=0}P\brak{\cond{X=0,Y=1}{X=0}}}{P\brak{X=0}P\brak{\cond{X=0,Y=1}{X=0}} + P\brak{X=1}P\brak{\cond{X=0,Y=1}{X=1}}} \\
     & = \frac{\frac{4}{7} \times \frac{2}{5}}{\frac{3}{7}\times\frac{1}{2}+\frac{4}{7}\times\frac{2}{5}}                   \\
     & = \frac{16}{31}
\end{align}
\end{document}
Assuming ball is not transferred,
\begin{align}
    P(X=0, Y=0) & = \frac{3}{7} \\
    P(X=1, Y=0) & = \frac{4}{7}
\end{align}

When the ball being transferred is red,
\begin{align}
    P\brak{\cond{X=0,Y=1}{X=0}} & = \frac{5}{10} \\
                                & =\frac{1}{2}
\end{align}

When the ball being transferred is black,
\begin{align}
    P\brak{\cond{X=0,Y=1}{X=1}} & = \frac{4}{10} \\
                                & =\frac{2}{5}
\end{align}

Now the probability of the transferred ball is black given drawn ball being red is

(According to Bayes' theorem)
\begin{align}
    P\brak{\cond{X=1}{X=0,Y=1}}
     & = \frac{P\brak{X=0}P\brak{\cond{X=0,Y=1}{X=0}}}{P\brak{X=0}P\brak{\cond{X=0,Y=1}{X=0}} + P\brak{X=1}P\brak{\cond{X=0,Y=1}{X=1}}} \\
     & = \frac{\frac{4}{7} \times \frac{2}{5}}{\frac{3}{7}\times\frac{1}{2}+\frac{4}{7}\times\frac{2}{5}}                   \\
     & = \frac{16}{31}
\end{align}
\end{document}
Assuming ball is not transferred,
\begin{align}
    P(X=0, Y=0) & = \frac{3}{7} \\
    P(X=1, Y=0) & = \frac{4}{7}
\end{align}

When the ball being transferred is red,
\begin{align}
    P\brak{\cond{X=0,Y=1}{X=0}} & = \frac{5}{10} \\
                                & =\frac{1}{2}
\end{align}

When the ball being transferred is black,
\begin{align}
    P\brak{\cond{X=0,Y=1}{X=1}} & = \frac{4}{10} \\
                                & =\frac{2}{5}
\end{align}

Now the probability of the transferred ball is black given drawn ball being red is

(According to Bayes' theorem)
\begin{align}
    P\brak{\cond{X=1}{X=0,Y=1}}
     & = \frac{P\brak{X=0}P\brak{\cond{X=0,Y=1}{X=0}}}{P\brak{X=0}P\brak{\cond{X=0,Y=1}{X=0}} + P\brak{X=1}P\brak{\cond{X=0,Y=1}{X=1}}} \\
     & = \frac{\frac{4}{7} \times \frac{2}{5}}{\frac{3}{7}\times\frac{1}{2}+\frac{4}{7}\times\frac{2}{5}}                   \\
     & = \frac{16}{31}
\end{align}
\end{document}
\section{Section 2}
\documentclass[journal,11pt,onecolumn]{IEEEtran}
\usepackage{setspace}
\usepackage{gensymb}
\singlespacing
\usepackage[cmex10]{amsmath}
\usepackage{amsthm}
\usepackage{mathrsfs}
\usepackage{txfonts}
\usepackage{stfloats}
\usepackage{bm}
\usepackage{cite}
\usepackage{cases}
\usepackage{subfig}
\usepackage{longtable}
\usepackage{multirow}
\usepackage{enumitem}
\usepackage{mathtools}
\usepackage{tikz}
\usepackage{circuitikz}
\usepackage{verbatim}
\usepackage[breaklinks=true]{hyperref}
\usepackage{tkz-euclide} % loads  TikZ and tkz-base
\usepackage{listings}
\usepackage{color}    
\usepackage{array}    
\usepackage{longtable}
\usepackage{calc}     
\usepackage{multirow} 
\usepackage{hhline}   
\usepackage{ifthen}   
\usepackage{lscape}     
\usepackage{chngcntr}
\usepackage{float}
\usepackage{gvv}

\begin{document}

\vspace{3cm}
\author{Ajay Krishnan K\\EE22BTECH11003}

\title{Assignment 2}
\maketitle

\textbf{Question 12.13.6.16}
\begin{enumerate}
    \item Bag I contains 3 red and 4 black balls and Bag II contains 4 red and 5 black balls.
          One ball is transferred from Bag I to Bag II and then a ball is drawn from Bag II.
          The ball so drawn is found to be red in colour. Find the probability that the
          transferred ball is black.
\end{enumerate}

\solution

\documentclass[journal,11pt,onecolumn]{IEEEtran}
\usepackage{setspace}
\usepackage{gensymb}
\singlespacing
\usepackage[cmex10]{amsmath}
\usepackage{amsthm}
\usepackage{mathrsfs}
\usepackage{txfonts}
\usepackage{stfloats}
\usepackage{bm}
\usepackage{cite}
\usepackage{cases}
\usepackage{subfig}
\usepackage{longtable}
\usepackage{multirow}
\usepackage{enumitem}
\usepackage{mathtools}
\usepackage{tikz}
\usepackage{circuitikz}
\usepackage{verbatim}
\usepackage[breaklinks=true]{hyperref}
\usepackage{tkz-euclide} % loads  TikZ and tkz-base
\usepackage{listings}
\usepackage{color}    
\usepackage{array}    
\usepackage{longtable}
\usepackage{calc}     
\usepackage{multirow} 
\usepackage{hhline}   
\usepackage{ifthen}   
\usepackage{lscape}     
\usepackage{chngcntr}
\usepackage{float}
\usepackage{gvv}

\begin{document}

\vspace{3cm}
\author{Ajay Krishnan K\\EE22BTECH11003}

\title{Assignment 2}
\maketitle

\textbf{Question 12.13.6.16}
\begin{enumerate}
    \item Bag I contains 3 red and 4 black balls and Bag II contains 4 red and 5 black balls.
          One ball is transferred from Bag I to Bag II and then a ball is drawn from Bag II.
          The ball so drawn is found to be red in colour. Find the probability that the
          transferred ball is black.
\end{enumerate}

\solution

\documentclass[journal,11pt,onecolumn]{IEEEtran}
\usepackage{setspace}
\usepackage{gensymb}
\singlespacing
\usepackage[cmex10]{amsmath}
\usepackage{amsthm}
\usepackage{mathrsfs}
\usepackage{txfonts}
\usepackage{stfloats}
\usepackage{bm}
\usepackage{cite}
\usepackage{cases}
\usepackage{subfig}
\usepackage{longtable}
\usepackage{multirow}
\usepackage{enumitem}
\usepackage{mathtools}
\usepackage{tikz}
\usepackage{circuitikz}
\usepackage{verbatim}
\usepackage[breaklinks=true]{hyperref}
\usepackage{tkz-euclide} % loads  TikZ and tkz-base
\usepackage{listings}
\usepackage{color}    
\usepackage{array}    
\usepackage{longtable}
\usepackage{calc}     
\usepackage{multirow} 
\usepackage{hhline}   
\usepackage{ifthen}   
\usepackage{lscape}     
\usepackage{chngcntr}
\usepackage{float}
\usepackage{gvv}

\begin{document}

\vspace{3cm}
\author{Ajay Krishnan K\\EE22BTECH11003}

\title{Assignment 2}
\maketitle

\textbf{Question 12.13.6.16}
\begin{enumerate}
    \item Bag I contains 3 red and 4 black balls and Bag II contains 4 red and 5 black balls.
          One ball is transferred from Bag I to Bag II and then a ball is drawn from Bag II.
          The ball so drawn is found to be red in colour. Find the probability that the
          transferred ball is black.
\end{enumerate}

\solution

\input{tables/main.tex}
Assuming ball is not transferred,
\begin{align}
    P(X=0, Y=0) & = \frac{3}{7} \\
    P(X=1, Y=0) & = \frac{4}{7}
\end{align}

When the ball being transferred is red,
\begin{align}
    P\brak{\cond{X=0,Y=1}{X=0}} & = \frac{5}{10} \\
                                & =\frac{1}{2}
\end{align}

When the ball being transferred is black,
\begin{align}
    P\brak{\cond{X=0,Y=1}{X=1}} & = \frac{4}{10} \\
                                & =\frac{2}{5}
\end{align}

Now the probability of the transferred ball is black given drawn ball being red is

(According to Bayes' theorem)
\begin{align}
    P\brak{\cond{X=1}{X=0,Y=1}}
     & = \frac{P\brak{X=0}P\brak{\cond{X=0,Y=1}{X=0}}}{P\brak{X=0}P\brak{\cond{X=0,Y=1}{X=0}} + P\brak{X=1}P\brak{\cond{X=0,Y=1}{X=1}}} \\
     & = \frac{\frac{4}{7} \times \frac{2}{5}}{\frac{3}{7}\times\frac{1}{2}+\frac{4}{7}\times\frac{2}{5}}                   \\
     & = \frac{16}{31}
\end{align}
\end{document}
Assuming ball is not transferred,
\begin{align}
    P(X=0, Y=0) & = \frac{3}{7} \\
    P(X=1, Y=0) & = \frac{4}{7}
\end{align}

When the ball being transferred is red,
\begin{align}
    P\brak{\cond{X=0,Y=1}{X=0}} & = \frac{5}{10} \\
                                & =\frac{1}{2}
\end{align}

When the ball being transferred is black,
\begin{align}
    P\brak{\cond{X=0,Y=1}{X=1}} & = \frac{4}{10} \\
                                & =\frac{2}{5}
\end{align}

Now the probability of the transferred ball is black given drawn ball being red is

(According to Bayes' theorem)
\begin{align}
    P\brak{\cond{X=1}{X=0,Y=1}}
     & = \frac{P\brak{X=0}P\brak{\cond{X=0,Y=1}{X=0}}}{P\brak{X=0}P\brak{\cond{X=0,Y=1}{X=0}} + P\brak{X=1}P\brak{\cond{X=0,Y=1}{X=1}}} \\
     & = \frac{\frac{4}{7} \times \frac{2}{5}}{\frac{3}{7}\times\frac{1}{2}+\frac{4}{7}\times\frac{2}{5}}                   \\
     & = \frac{16}{31}
\end{align}
\end{document}
Assuming ball is not transferred,
\begin{align}
    P(X=0, Y=0) & = \frac{3}{7} \\
    P(X=1, Y=0) & = \frac{4}{7}
\end{align}

When the ball being transferred is red,
\begin{align}
    P\brak{\cond{X=0,Y=1}{X=0}} & = \frac{5}{10} \\
                                & =\frac{1}{2}
\end{align}

When the ball being transferred is black,
\begin{align}
    P\brak{\cond{X=0,Y=1}{X=1}} & = \frac{4}{10} \\
                                & =\frac{2}{5}
\end{align}

Now the probability of the transferred ball is black given drawn ball being red is

(According to Bayes' theorem)
\begin{align}
    P\brak{\cond{X=1}{X=0,Y=1}}
     & = \frac{P\brak{X=0}P\brak{\cond{X=0,Y=1}{X=0}}}{P\brak{X=0}P\brak{\cond{X=0,Y=1}{X=0}} + P\brak{X=1}P\brak{\cond{X=0,Y=1}{X=1}}} \\
     & = \frac{\frac{4}{7} \times \frac{2}{5}}{\frac{3}{7}\times\frac{1}{2}+\frac{4}{7}\times\frac{2}{5}}                   \\
     & = \frac{16}{31}
\end{align}
\end{document}
\section{Section 3}
\documentclass[journal,11pt,onecolumn]{IEEEtran}
\usepackage{setspace}
\usepackage{gensymb}
\singlespacing
\usepackage[cmex10]{amsmath}
\usepackage{amsthm}
\usepackage{mathrsfs}
\usepackage{txfonts}
\usepackage{stfloats}
\usepackage{bm}
\usepackage{cite}
\usepackage{cases}
\usepackage{subfig}
\usepackage{longtable}
\usepackage{multirow}
\usepackage{enumitem}
\usepackage{mathtools}
\usepackage{tikz}
\usepackage{circuitikz}
\usepackage{verbatim}
\usepackage[breaklinks=true]{hyperref}
\usepackage{tkz-euclide} % loads  TikZ and tkz-base
\usepackage{listings}
\usepackage{color}    
\usepackage{array}    
\usepackage{longtable}
\usepackage{calc}     
\usepackage{multirow} 
\usepackage{hhline}   
\usepackage{ifthen}   
\usepackage{lscape}     
\usepackage{chngcntr}
\usepackage{float}
\usepackage{gvv}

\begin{document}

\vspace{3cm}
\author{Ajay Krishnan K\\EE22BTECH11003}

\title{Assignment 2}
\maketitle

\textbf{Question 12.13.6.16}
\begin{enumerate}
    \item Bag I contains 3 red and 4 black balls and Bag II contains 4 red and 5 black balls.
          One ball is transferred from Bag I to Bag II and then a ball is drawn from Bag II.
          The ball so drawn is found to be red in colour. Find the probability that the
          transferred ball is black.
\end{enumerate}

\solution

\documentclass[journal,11pt,onecolumn]{IEEEtran}
\usepackage{setspace}
\usepackage{gensymb}
\singlespacing
\usepackage[cmex10]{amsmath}
\usepackage{amsthm}
\usepackage{mathrsfs}
\usepackage{txfonts}
\usepackage{stfloats}
\usepackage{bm}
\usepackage{cite}
\usepackage{cases}
\usepackage{subfig}
\usepackage{longtable}
\usepackage{multirow}
\usepackage{enumitem}
\usepackage{mathtools}
\usepackage{tikz}
\usepackage{circuitikz}
\usepackage{verbatim}
\usepackage[breaklinks=true]{hyperref}
\usepackage{tkz-euclide} % loads  TikZ and tkz-base
\usepackage{listings}
\usepackage{color}    
\usepackage{array}    
\usepackage{longtable}
\usepackage{calc}     
\usepackage{multirow} 
\usepackage{hhline}   
\usepackage{ifthen}   
\usepackage{lscape}     
\usepackage{chngcntr}
\usepackage{float}
\usepackage{gvv}

\begin{document}

\vspace{3cm}
\author{Ajay Krishnan K\\EE22BTECH11003}

\title{Assignment 2}
\maketitle

\textbf{Question 12.13.6.16}
\begin{enumerate}
    \item Bag I contains 3 red and 4 black balls and Bag II contains 4 red and 5 black balls.
          One ball is transferred from Bag I to Bag II and then a ball is drawn from Bag II.
          The ball so drawn is found to be red in colour. Find the probability that the
          transferred ball is black.
\end{enumerate}

\solution

\documentclass[journal,11pt,onecolumn]{IEEEtran}
\usepackage{setspace}
\usepackage{gensymb}
\singlespacing
\usepackage[cmex10]{amsmath}
\usepackage{amsthm}
\usepackage{mathrsfs}
\usepackage{txfonts}
\usepackage{stfloats}
\usepackage{bm}
\usepackage{cite}
\usepackage{cases}
\usepackage{subfig}
\usepackage{longtable}
\usepackage{multirow}
\usepackage{enumitem}
\usepackage{mathtools}
\usepackage{tikz}
\usepackage{circuitikz}
\usepackage{verbatim}
\usepackage[breaklinks=true]{hyperref}
\usepackage{tkz-euclide} % loads  TikZ and tkz-base
\usepackage{listings}
\usepackage{color}    
\usepackage{array}    
\usepackage{longtable}
\usepackage{calc}     
\usepackage{multirow} 
\usepackage{hhline}   
\usepackage{ifthen}   
\usepackage{lscape}     
\usepackage{chngcntr}
\usepackage{float}
\usepackage{gvv}

\begin{document}

\vspace{3cm}
\author{Ajay Krishnan K\\EE22BTECH11003}

\title{Assignment 2}
\maketitle

\textbf{Question 12.13.6.16}
\begin{enumerate}
    \item Bag I contains 3 red and 4 black balls and Bag II contains 4 red and 5 black balls.
          One ball is transferred from Bag I to Bag II and then a ball is drawn from Bag II.
          The ball so drawn is found to be red in colour. Find the probability that the
          transferred ball is black.
\end{enumerate}

\solution

\input{tables/main.tex}
Assuming ball is not transferred,
\begin{align}
    P(X=0, Y=0) & = \frac{3}{7} \\
    P(X=1, Y=0) & = \frac{4}{7}
\end{align}

When the ball being transferred is red,
\begin{align}
    P\brak{\cond{X=0,Y=1}{X=0}} & = \frac{5}{10} \\
                                & =\frac{1}{2}
\end{align}

When the ball being transferred is black,
\begin{align}
    P\brak{\cond{X=0,Y=1}{X=1}} & = \frac{4}{10} \\
                                & =\frac{2}{5}
\end{align}

Now the probability of the transferred ball is black given drawn ball being red is

(According to Bayes' theorem)
\begin{align}
    P\brak{\cond{X=1}{X=0,Y=1}}
     & = \frac{P\brak{X=0}P\brak{\cond{X=0,Y=1}{X=0}}}{P\brak{X=0}P\brak{\cond{X=0,Y=1}{X=0}} + P\brak{X=1}P\brak{\cond{X=0,Y=1}{X=1}}} \\
     & = \frac{\frac{4}{7} \times \frac{2}{5}}{\frac{3}{7}\times\frac{1}{2}+\frac{4}{7}\times\frac{2}{5}}                   \\
     & = \frac{16}{31}
\end{align}
\end{document}
Assuming ball is not transferred,
\begin{align}
    P(X=0, Y=0) & = \frac{3}{7} \\
    P(X=1, Y=0) & = \frac{4}{7}
\end{align}

When the ball being transferred is red,
\begin{align}
    P\brak{\cond{X=0,Y=1}{X=0}} & = \frac{5}{10} \\
                                & =\frac{1}{2}
\end{align}

When the ball being transferred is black,
\begin{align}
    P\brak{\cond{X=0,Y=1}{X=1}} & = \frac{4}{10} \\
                                & =\frac{2}{5}
\end{align}

Now the probability of the transferred ball is black given drawn ball being red is

(According to Bayes' theorem)
\begin{align}
    P\brak{\cond{X=1}{X=0,Y=1}}
     & = \frac{P\brak{X=0}P\brak{\cond{X=0,Y=1}{X=0}}}{P\brak{X=0}P\brak{\cond{X=0,Y=1}{X=0}} + P\brak{X=1}P\brak{\cond{X=0,Y=1}{X=1}}} \\
     & = \frac{\frac{4}{7} \times \frac{2}{5}}{\frac{3}{7}\times\frac{1}{2}+\frac{4}{7}\times\frac{2}{5}}                   \\
     & = \frac{16}{31}
\end{align}
\end{document}
Assuming ball is not transferred,
\begin{align}
    P(X=0, Y=0) & = \frac{3}{7} \\
    P(X=1, Y=0) & = \frac{4}{7}
\end{align}

When the ball being transferred is red,
\begin{align}
    P\brak{\cond{X=0,Y=1}{X=0}} & = \frac{5}{10} \\
                                & =\frac{1}{2}
\end{align}

When the ball being transferred is black,
\begin{align}
    P\brak{\cond{X=0,Y=1}{X=1}} & = \frac{4}{10} \\
                                & =\frac{2}{5}
\end{align}

Now the probability of the transferred ball is black given drawn ball being red is

(According to Bayes' theorem)
\begin{align}
    P\brak{\cond{X=1}{X=0,Y=1}}
     & = \frac{P\brak{X=0}P\brak{\cond{X=0,Y=1}{X=0}}}{P\brak{X=0}P\brak{\cond{X=0,Y=1}{X=0}} + P\brak{X=1}P\brak{\cond{X=0,Y=1}{X=1}}} \\
     & = \frac{\frac{4}{7} \times \frac{2}{5}}{\frac{3}{7}\times\frac{1}{2}+\frac{4}{7}\times\frac{2}{5}}                   \\
     & = \frac{16}{31}
\end{align}
\end{document}
\section{Section 4}
\documentclass[journal,11pt,onecolumn]{IEEEtran}
\usepackage{setspace}
\usepackage{gensymb}
\singlespacing
\usepackage[cmex10]{amsmath}
\usepackage{amsthm}
\usepackage{mathrsfs}
\usepackage{txfonts}
\usepackage{stfloats}
\usepackage{bm}
\usepackage{cite}
\usepackage{cases}
\usepackage{subfig}
\usepackage{longtable}
\usepackage{multirow}
\usepackage{enumitem}
\usepackage{mathtools}
\usepackage{tikz}
\usepackage{circuitikz}
\usepackage{verbatim}
\usepackage[breaklinks=true]{hyperref}
\usepackage{tkz-euclide} % loads  TikZ and tkz-base
\usepackage{listings}
\usepackage{color}    
\usepackage{array}    
\usepackage{longtable}
\usepackage{calc}     
\usepackage{multirow} 
\usepackage{hhline}   
\usepackage{ifthen}   
\usepackage{lscape}     
\usepackage{chngcntr}
\usepackage{float}
\usepackage{gvv}

\begin{document}

\vspace{3cm}
\author{Ajay Krishnan K\\EE22BTECH11003}

\title{Assignment 2}
\maketitle

\textbf{Question 12.13.6.16}
\begin{enumerate}
    \item Bag I contains 3 red and 4 black balls and Bag II contains 4 red and 5 black balls.
          One ball is transferred from Bag I to Bag II and then a ball is drawn from Bag II.
          The ball so drawn is found to be red in colour. Find the probability that the
          transferred ball is black.
\end{enumerate}

\solution

\documentclass[journal,11pt,onecolumn]{IEEEtran}
\usepackage{setspace}
\usepackage{gensymb}
\singlespacing
\usepackage[cmex10]{amsmath}
\usepackage{amsthm}
\usepackage{mathrsfs}
\usepackage{txfonts}
\usepackage{stfloats}
\usepackage{bm}
\usepackage{cite}
\usepackage{cases}
\usepackage{subfig}
\usepackage{longtable}
\usepackage{multirow}
\usepackage{enumitem}
\usepackage{mathtools}
\usepackage{tikz}
\usepackage{circuitikz}
\usepackage{verbatim}
\usepackage[breaklinks=true]{hyperref}
\usepackage{tkz-euclide} % loads  TikZ and tkz-base
\usepackage{listings}
\usepackage{color}    
\usepackage{array}    
\usepackage{longtable}
\usepackage{calc}     
\usepackage{multirow} 
\usepackage{hhline}   
\usepackage{ifthen}   
\usepackage{lscape}     
\usepackage{chngcntr}
\usepackage{float}
\usepackage{gvv}

\begin{document}

\vspace{3cm}
\author{Ajay Krishnan K\\EE22BTECH11003}

\title{Assignment 2}
\maketitle

\textbf{Question 12.13.6.16}
\begin{enumerate}
    \item Bag I contains 3 red and 4 black balls and Bag II contains 4 red and 5 black balls.
          One ball is transferred from Bag I to Bag II and then a ball is drawn from Bag II.
          The ball so drawn is found to be red in colour. Find the probability that the
          transferred ball is black.
\end{enumerate}

\solution

\documentclass[journal,11pt,onecolumn]{IEEEtran}
\usepackage{setspace}
\usepackage{gensymb}
\singlespacing
\usepackage[cmex10]{amsmath}
\usepackage{amsthm}
\usepackage{mathrsfs}
\usepackage{txfonts}
\usepackage{stfloats}
\usepackage{bm}
\usepackage{cite}
\usepackage{cases}
\usepackage{subfig}
\usepackage{longtable}
\usepackage{multirow}
\usepackage{enumitem}
\usepackage{mathtools}
\usepackage{tikz}
\usepackage{circuitikz}
\usepackage{verbatim}
\usepackage[breaklinks=true]{hyperref}
\usepackage{tkz-euclide} % loads  TikZ and tkz-base
\usepackage{listings}
\usepackage{color}    
\usepackage{array}    
\usepackage{longtable}
\usepackage{calc}     
\usepackage{multirow} 
\usepackage{hhline}   
\usepackage{ifthen}   
\usepackage{lscape}     
\usepackage{chngcntr}
\usepackage{float}
\usepackage{gvv}

\begin{document}

\vspace{3cm}
\author{Ajay Krishnan K\\EE22BTECH11003}

\title{Assignment 2}
\maketitle

\textbf{Question 12.13.6.16}
\begin{enumerate}
    \item Bag I contains 3 red and 4 black balls and Bag II contains 4 red and 5 black balls.
          One ball is transferred from Bag I to Bag II and then a ball is drawn from Bag II.
          The ball so drawn is found to be red in colour. Find the probability that the
          transferred ball is black.
\end{enumerate}

\solution

\input{tables/main.tex}
Assuming ball is not transferred,
\begin{align}
    P(X=0, Y=0) & = \frac{3}{7} \\
    P(X=1, Y=0) & = \frac{4}{7}
\end{align}

When the ball being transferred is red,
\begin{align}
    P\brak{\cond{X=0,Y=1}{X=0}} & = \frac{5}{10} \\
                                & =\frac{1}{2}
\end{align}

When the ball being transferred is black,
\begin{align}
    P\brak{\cond{X=0,Y=1}{X=1}} & = \frac{4}{10} \\
                                & =\frac{2}{5}
\end{align}

Now the probability of the transferred ball is black given drawn ball being red is

(According to Bayes' theorem)
\begin{align}
    P\brak{\cond{X=1}{X=0,Y=1}}
     & = \frac{P\brak{X=0}P\brak{\cond{X=0,Y=1}{X=0}}}{P\brak{X=0}P\brak{\cond{X=0,Y=1}{X=0}} + P\brak{X=1}P\brak{\cond{X=0,Y=1}{X=1}}} \\
     & = \frac{\frac{4}{7} \times \frac{2}{5}}{\frac{3}{7}\times\frac{1}{2}+\frac{4}{7}\times\frac{2}{5}}                   \\
     & = \frac{16}{31}
\end{align}
\end{document}
Assuming ball is not transferred,
\begin{align}
    P(X=0, Y=0) & = \frac{3}{7} \\
    P(X=1, Y=0) & = \frac{4}{7}
\end{align}

When the ball being transferred is red,
\begin{align}
    P\brak{\cond{X=0,Y=1}{X=0}} & = \frac{5}{10} \\
                                & =\frac{1}{2}
\end{align}

When the ball being transferred is black,
\begin{align}
    P\brak{\cond{X=0,Y=1}{X=1}} & = \frac{4}{10} \\
                                & =\frac{2}{5}
\end{align}

Now the probability of the transferred ball is black given drawn ball being red is

(According to Bayes' theorem)
\begin{align}
    P\brak{\cond{X=1}{X=0,Y=1}}
     & = \frac{P\brak{X=0}P\brak{\cond{X=0,Y=1}{X=0}}}{P\brak{X=0}P\brak{\cond{X=0,Y=1}{X=0}} + P\brak{X=1}P\brak{\cond{X=0,Y=1}{X=1}}} \\
     & = \frac{\frac{4}{7} \times \frac{2}{5}}{\frac{3}{7}\times\frac{1}{2}+\frac{4}{7}\times\frac{2}{5}}                   \\
     & = \frac{16}{31}
\end{align}
\end{document}
Assuming ball is not transferred,
\begin{align}
    P(X=0, Y=0) & = \frac{3}{7} \\
    P(X=1, Y=0) & = \frac{4}{7}
\end{align}

When the ball being transferred is red,
\begin{align}
    P\brak{\cond{X=0,Y=1}{X=0}} & = \frac{5}{10} \\
                                & =\frac{1}{2}
\end{align}

When the ball being transferred is black,
\begin{align}
    P\brak{\cond{X=0,Y=1}{X=1}} & = \frac{4}{10} \\
                                & =\frac{2}{5}
\end{align}

Now the probability of the transferred ball is black given drawn ball being red is

(According to Bayes' theorem)
\begin{align}
    P\brak{\cond{X=1}{X=0,Y=1}}
     & = \frac{P\brak{X=0}P\brak{\cond{X=0,Y=1}{X=0}}}{P\brak{X=0}P\brak{\cond{X=0,Y=1}{X=0}} + P\brak{X=1}P\brak{\cond{X=0,Y=1}{X=1}}} \\
     & = \frac{\frac{4}{7} \times \frac{2}{5}}{\frac{3}{7}\times\frac{1}{2}+\frac{4}{7}\times\frac{2}{5}}                   \\
     & = \frac{16}{31}
\end{align}
\end{document}
\section{Section 5}
\documentclass[journal,11pt,onecolumn]{IEEEtran}
\usepackage{setspace}
\usepackage{gensymb}
\singlespacing
\usepackage[cmex10]{amsmath}
\usepackage{amsthm}
\usepackage{mathrsfs}
\usepackage{txfonts}
\usepackage{stfloats}
\usepackage{bm}
\usepackage{cite}
\usepackage{cases}
\usepackage{subfig}
\usepackage{longtable}
\usepackage{multirow}
\usepackage{enumitem}
\usepackage{mathtools}
\usepackage{tikz}
\usepackage{circuitikz}
\usepackage{verbatim}
\usepackage[breaklinks=true]{hyperref}
\usepackage{tkz-euclide} % loads  TikZ and tkz-base
\usepackage{listings}
\usepackage{color}    
\usepackage{array}    
\usepackage{longtable}
\usepackage{calc}     
\usepackage{multirow} 
\usepackage{hhline}   
\usepackage{ifthen}   
\usepackage{lscape}     
\usepackage{chngcntr}
\usepackage{float}
\usepackage{gvv}

\begin{document}

\vspace{3cm}
\author{Ajay Krishnan K\\EE22BTECH11003}

\title{Assignment 2}
\maketitle

\textbf{Question 12.13.6.16}
\begin{enumerate}
    \item Bag I contains 3 red and 4 black balls and Bag II contains 4 red and 5 black balls.
          One ball is transferred from Bag I to Bag II and then a ball is drawn from Bag II.
          The ball so drawn is found to be red in colour. Find the probability that the
          transferred ball is black.
\end{enumerate}

\solution

\documentclass[journal,11pt,onecolumn]{IEEEtran}
\usepackage{setspace}
\usepackage{gensymb}
\singlespacing
\usepackage[cmex10]{amsmath}
\usepackage{amsthm}
\usepackage{mathrsfs}
\usepackage{txfonts}
\usepackage{stfloats}
\usepackage{bm}
\usepackage{cite}
\usepackage{cases}
\usepackage{subfig}
\usepackage{longtable}
\usepackage{multirow}
\usepackage{enumitem}
\usepackage{mathtools}
\usepackage{tikz}
\usepackage{circuitikz}
\usepackage{verbatim}
\usepackage[breaklinks=true]{hyperref}
\usepackage{tkz-euclide} % loads  TikZ and tkz-base
\usepackage{listings}
\usepackage{color}    
\usepackage{array}    
\usepackage{longtable}
\usepackage{calc}     
\usepackage{multirow} 
\usepackage{hhline}   
\usepackage{ifthen}   
\usepackage{lscape}     
\usepackage{chngcntr}
\usepackage{float}
\usepackage{gvv}

\begin{document}

\vspace{3cm}
\author{Ajay Krishnan K\\EE22BTECH11003}

\title{Assignment 2}
\maketitle

\textbf{Question 12.13.6.16}
\begin{enumerate}
    \item Bag I contains 3 red and 4 black balls and Bag II contains 4 red and 5 black balls.
          One ball is transferred from Bag I to Bag II and then a ball is drawn from Bag II.
          The ball so drawn is found to be red in colour. Find the probability that the
          transferred ball is black.
\end{enumerate}

\solution

\documentclass[journal,11pt,onecolumn]{IEEEtran}
\usepackage{setspace}
\usepackage{gensymb}
\singlespacing
\usepackage[cmex10]{amsmath}
\usepackage{amsthm}
\usepackage{mathrsfs}
\usepackage{txfonts}
\usepackage{stfloats}
\usepackage{bm}
\usepackage{cite}
\usepackage{cases}
\usepackage{subfig}
\usepackage{longtable}
\usepackage{multirow}
\usepackage{enumitem}
\usepackage{mathtools}
\usepackage{tikz}
\usepackage{circuitikz}
\usepackage{verbatim}
\usepackage[breaklinks=true]{hyperref}
\usepackage{tkz-euclide} % loads  TikZ and tkz-base
\usepackage{listings}
\usepackage{color}    
\usepackage{array}    
\usepackage{longtable}
\usepackage{calc}     
\usepackage{multirow} 
\usepackage{hhline}   
\usepackage{ifthen}   
\usepackage{lscape}     
\usepackage{chngcntr}
\usepackage{float}
\usepackage{gvv}

\begin{document}

\vspace{3cm}
\author{Ajay Krishnan K\\EE22BTECH11003}

\title{Assignment 2}
\maketitle

\textbf{Question 12.13.6.16}
\begin{enumerate}
    \item Bag I contains 3 red and 4 black balls and Bag II contains 4 red and 5 black balls.
          One ball is transferred from Bag I to Bag II and then a ball is drawn from Bag II.
          The ball so drawn is found to be red in colour. Find the probability that the
          transferred ball is black.
\end{enumerate}

\solution

\input{tables/main.tex}
Assuming ball is not transferred,
\begin{align}
    P(X=0, Y=0) & = \frac{3}{7} \\
    P(X=1, Y=0) & = \frac{4}{7}
\end{align}

When the ball being transferred is red,
\begin{align}
    P\brak{\cond{X=0,Y=1}{X=0}} & = \frac{5}{10} \\
                                & =\frac{1}{2}
\end{align}

When the ball being transferred is black,
\begin{align}
    P\brak{\cond{X=0,Y=1}{X=1}} & = \frac{4}{10} \\
                                & =\frac{2}{5}
\end{align}

Now the probability of the transferred ball is black given drawn ball being red is

(According to Bayes' theorem)
\begin{align}
    P\brak{\cond{X=1}{X=0,Y=1}}
     & = \frac{P\brak{X=0}P\brak{\cond{X=0,Y=1}{X=0}}}{P\brak{X=0}P\brak{\cond{X=0,Y=1}{X=0}} + P\brak{X=1}P\brak{\cond{X=0,Y=1}{X=1}}} \\
     & = \frac{\frac{4}{7} \times \frac{2}{5}}{\frac{3}{7}\times\frac{1}{2}+\frac{4}{7}\times\frac{2}{5}}                   \\
     & = \frac{16}{31}
\end{align}
\end{document}
Assuming ball is not transferred,
\begin{align}
    P(X=0, Y=0) & = \frac{3}{7} \\
    P(X=1, Y=0) & = \frac{4}{7}
\end{align}

When the ball being transferred is red,
\begin{align}
    P\brak{\cond{X=0,Y=1}{X=0}} & = \frac{5}{10} \\
                                & =\frac{1}{2}
\end{align}

When the ball being transferred is black,
\begin{align}
    P\brak{\cond{X=0,Y=1}{X=1}} & = \frac{4}{10} \\
                                & =\frac{2}{5}
\end{align}

Now the probability of the transferred ball is black given drawn ball being red is

(According to Bayes' theorem)
\begin{align}
    P\brak{\cond{X=1}{X=0,Y=1}}
     & = \frac{P\brak{X=0}P\brak{\cond{X=0,Y=1}{X=0}}}{P\brak{X=0}P\brak{\cond{X=0,Y=1}{X=0}} + P\brak{X=1}P\brak{\cond{X=0,Y=1}{X=1}}} \\
     & = \frac{\frac{4}{7} \times \frac{2}{5}}{\frac{3}{7}\times\frac{1}{2}+\frac{4}{7}\times\frac{2}{5}}                   \\
     & = \frac{16}{31}
\end{align}
\end{document}
Assuming ball is not transferred,
\begin{align}
    P(X=0, Y=0) & = \frac{3}{7} \\
    P(X=1, Y=0) & = \frac{4}{7}
\end{align}

When the ball being transferred is red,
\begin{align}
    P\brak{\cond{X=0,Y=1}{X=0}} & = \frac{5}{10} \\
                                & =\frac{1}{2}
\end{align}

When the ball being transferred is black,
\begin{align}
    P\brak{\cond{X=0,Y=1}{X=1}} & = \frac{4}{10} \\
                                & =\frac{2}{5}
\end{align}

Now the probability of the transferred ball is black given drawn ball being red is

(According to Bayes' theorem)
\begin{align}
    P\brak{\cond{X=1}{X=0,Y=1}}
     & = \frac{P\brak{X=0}P\brak{\cond{X=0,Y=1}{X=0}}}{P\brak{X=0}P\brak{\cond{X=0,Y=1}{X=0}} + P\brak{X=1}P\brak{\cond{X=0,Y=1}{X=1}}} \\
     & = \frac{\frac{4}{7} \times \frac{2}{5}}{\frac{3}{7}\times\frac{1}{2}+\frac{4}{7}\times\frac{2}{5}}                   \\
     & = \frac{16}{31}
\end{align}
\end{document}

\end{document}